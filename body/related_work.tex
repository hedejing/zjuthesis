\chapter{背景知识与相关工作}

\section{线绘制}

在一些传统的艺术效果中,例如笔墨风格和技术说明书风格,线条的绘制是很重要的一部分。为了实现这类艺术效果,计算机图形学的研究者们提出了多种不同的算法来实现基于三维模型的线绘制。根据视点相关性,可以将线绘制所讨论的线条类型分为\vdl{}(view-dependent lines)和\vidl{}(view-independent lines)。

在\vdl{}中,\con{}是最为常用的线条类型。简单来说,\con{}表示的就是一般意义上物体的“轮廓”。在计算机图形学领域中,常用的三维模型由三角形网格构成,而\con{}会出现在正对视点和反对视点的两个相邻三角形上。\scon{}也是一种\vdl{},它是对\con{}的补充和延展。

\vidl{}也是表现几何体表面的重要特征,折痕(crease)就是一种常见的\vidl{}。不过由于它们的出现和视点无关,所以它们自然是\stc{},本文不会对它们进行进一步的讨论。

\section{\stc{}\npr{}}

线绘制属于\npr{}技术中的一种。一般而言,\npr{}的相关研究只关注如何解决单目绘制下的问题。随着近年来虚拟现实技术的发展,\npr{}和双目绘制的结合越来越受到关注,出现了不少关于\npr{}在双目绘制下的问题的研究。为了避免双目绘制下基于笔画的图像风格化算法\cite{hertzmann1998painterly}带来的问题,\citeauthor{northam2012consistent}\cite{northam2012consistent,northam2013stereoscopic}提出了将左眼和右眼的画面分离成离散的差异层的方法。在得到左眼和右眼对应的差异层后,他们的方法将对应的层组合起来再作进一步的基于笔画的风格化,从而得到\stc{}的结果。

除了基于图像的风格化算法,基于三维模型的线绘制在双目绘制下的问题也有一定的研究。\citeauthor{kim2013stereoscopic}在他们的工作中具体定义了关于轮廓线的\stc{}概念,并提出了在多个视点上检查对极曲线上的对应轮廓点的的方法来保证轮廓线的立体一致性。此外,\citeauthor{bukenberger2018stereo}在近年提出了一个新的方法\cite{bukenberger2018stereo}来解决\stc{}轮廓线这个问题。与前者在图像空间中处理的方法相比,他们的方法着力在利用物体空间的信息来解决这个问题。在他们的方法中,首先要绘制出不同给定视点下的轮廓线的结果并以物体空间的形式存储下来,接着利用这些信息通过插值得出新视点下立体一致的轮廓线的画面。

在\stc{}的轮廓线绘制的基础上,如何做进一步的风格化的问题也得到了关注\cite{northrup2000artistic,kalnins2003coherent} 。\citeauthor{kim2013stereoscopic}和\citeauthor{bukenberger2018stereo}都在他们各自的工作中提出了对\stc{}的轮廓线进行风格化的方法。在获得不同视点下轮廓点的对应关系的基础上,\citeauthor{kim2013stereoscopic}通过在不同视点中传递风格化参数的方法来保证进一步的风格化处理依然保持最后结果是\stc{}。在\citeauthor{bukenberger2018stereo}的工作中,进一步的风格化参数由物体的三维表面的一些特征决定,以此来保证结果的时序一致性(temporal coherency)。同样,本文将先就如何实现\stc{}\con{}和\scon{}来进行阐述,然后再描述进一步风格化的解决方法。

\section{\ppll{}}

链表是计算机领域中常用的一种数据结构。链表往往用于存储不定长度的数据集合,在各式各样的CPU上的算法上很常见也易于实现,但是在GPU上没有简单直接的实现方法。\citeauthor{yang2010real}提出了一个高效的方法\cite{yang2010real}来在GPU上构建链表。他们的方法使用一个缓冲来存储所有的链表节点,并使用另一个图像空间的缓冲区来存储逐个像素上的链表头部节点,这些头部节点指向上述第一个缓冲中的用于存储实际信息的节点。\ppll{}是用来处理逐个像素上的多个片段(fragments)的最具有普适性的工具。为了避免轮廓点的重叠带来的问题,本文设计的方法使\ppll{}来访问位于同一个像素点上的多个不同轮廓点的数据。由于本文设计的方法只需要对轮廓点建立链表,而这些轮廓点只占整体画面的一小部分,所以\ppll{}的内存占用量并不大。