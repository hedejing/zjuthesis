\cleardoublepage
\chapternonum{摘要}
线绘制是一种重要且精确的表现物体外形的方法。作为线绘制和立体绘制的组合,立体线绘制不仅高效地表现了外形,还提供给用户一种立体三维世界的视觉体验。轮廓线是最为重要的一种绘制线。但是,由于其视点相关的特性,轮廓线必须要以对两眼一致的方式进行绘制,否则会导致双目竞争以及视觉效果上的不适。本文提出了一种新的\stc{}轮廓线的实时绘制方法。首先,我们拓展了\epsl{}这一概念,并推导出了一种基于轮廓点的对应视点的轨迹单调性的标准来判断\epsl{}。然后,区别于前人采样多个视点的方法,我们设计了一种图像空间的搜索算法来测试轮廓点的\epsl{}。实验结果表面本文所提出的算法的计算耗时远小于前人的工作所需要的计算耗时,因此能够实现\stc{}轮廓线的实时绘制和编辑,例如改变摄像机的视点位置,修改物体的几何,调整参数从而展示轮廓线的不同细节等等。


\cleardoublepage
\chapternonum{Abstract}
Line drawing is an important and concise method to depict the shape of an object. Stereo line drawing, a combination of line drawing and stereo rendering, not only efficiently conveys shape but also provides users with a visual experience of a stereoscopic 3D world. Contours are the most important lines to draw. However, contours must be rendered consistently for two eyes because of their view-dependent nature; otherwise, they cause binocular rivalry and viewing discomfort. This paper proposes a novel solution to draw stereo-consistent contours in real time. First, we extend the concept of epipolar-slidability and derive a new criterion to check epipolar-slidability by the monotonicity of the trajectory of the viewpoints of contour points. Then, we design an algorithm to test the epipolar-slidability of contours by conducting an image space search rather than sampling multiple viewpoints. Results show that the proposed method has a much lower cost than that of previous works, therefore enables the real-time rendering and editing of stereo-consistent contours for users, such as changing camera viewpoints, editing object geometry, tweaking parameters to show contours with different details, etc.